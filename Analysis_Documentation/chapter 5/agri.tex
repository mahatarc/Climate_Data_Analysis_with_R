\chapter{Merging with Agriculture Dataset}

Imagine you are a farmer or an agricultural planner trying to make the best decisions for the upcoming planting season. What if you could peek into the climate’s behavior — temperature changes, rainfall patterns, wind speed — and understand exactly how these factors affect crop growth?

In this chapter, we take a fascinating step forward: we merge agricultural data with climate data. By combining these two worlds, we can uncover deeper insights about how weather and climate influence agricultural productivity, helping farmers, scientists, and policymakers make smarter, data-driven decisions. Let’s think about this together:

\begin{itemize}
    \item How do you think temperature fluctuations affect crop yield?
    \item What role does rainfall play in the growth cycle of different crops?
    \item Can wind patterns influence soil erosion or pollination?
\end{itemize}

By the end of this chapter, you’ll not only learn how to merge these datasets technically, but also explore why this integration is so crucial for understanding the bigger picture of agriculture under changing climate conditions.

Ready to see the magic happen when data meets reality? Let’s dive in!

\subsection*{Sourcing Agricultural Data}
To carry out this integration, we sourced agricultural data from a publicly available interactive visualization platform: \\
\href{https://public.tableau.com/app/profile/sadichchha1369/viz/NepaCropMapwithprovinceSAMPLE_15636442141840/Dashboard1}{Tableau Public - Nepal Crop Map Dashboard}\\
This dataset includes district-level agricultural information, which we align with our climate dataset to explore temporal and spatial patterns in greater depth.
But before merging, let’s take a step back. Data, especially from real-world sources is rarely perfect. It often contains missing entries, inconsistent formats, or irrelevant information. That’s why our first task is to clean the agricultural dataset to ensure it aligns smoothly with our climate data.


\section{Data Preparation}

Cleaning is a critical step in any data analysis pipeline. Think of it as preparing a field before planting .We remove the weeds, till the soil, and make sure everything is ready for growth. Similarly, we clean our dataset to remove inconsistencies and get it analysis-ready.

\subsection*{Loading Dataset}
To load the agricultural dataset into R, we use the following command:

\begin{verbatim}
data_agri <- read.csv("/content/agriculture.csv")
\end{verbatim}

\subsection*{Data Inspection}
With the agricultural dataset already loaded into \texttt{data\_agri}, our next step is to inspect its structure and identify any data quality issues. This inspection helps us understand what kind of cleaning might be necessary before merging it with the climate dataset. We use the following commands to explore the dataset:

\begin{verbatim}
glimpse(data_agri)
\end{verbatim}

% Figure here-----------------------------
\begin{figure}[h]
\centering
\includegraphics[width=0.6\textwidth]{figures/agri_glimpse.jpg}
\caption{Glimpse of Agricultural Dataset}
\end{figure}

\subsection*{Data Cleaning}
\subsubsection*{Checking for Missing Values}
The next step is checking for missing values in the dataset to understand the completeness of the data.

\begin{verbatim}
sum(is.na(data_agri))          # Total missing values
colSums(is.na(data_agri))      # Missing values per column
\end{verbatim}

% Figure here------------------------
\begin{figure}[h]
\centering
\includegraphics[width=0.8\textwidth]{figures/missing_data_agri.jpg}
\caption{Missing data information
}

\end{figure}

\textbf{Explanation of each function:}
\begin{itemize}
    \item \texttt{glimpse(data\_agri)}: Offers a compact view of the dataset, showing each column’s name, data type, and a preview of its values.
    \item \texttt{sum(is.na(data\_agri))}: Displays the total count of missing values across the entire dataset.
    \item \texttt{colSums(is.na(data\_agri))}: Shows the number of missing entries in each column individually, helping us pinpoint where problems lie.
\end{itemize}

This step is crucial for understanding the completeness and structure of the data. Based on these results, we’ll decide which cleaning operations are necessary to prepare the dataset for merging with the climate records.

\subsubsection*{Dropping Unwanted Columns}
Based on our inspection, we identified several metadata and geometry-related columns in the agricultural dataset that are not relevant for our analysis. These include administrative codes, geometry information, shape metrics, and other auxiliary fields. To streamline our dataset and retain only meaningful columns, we define a list of columns to drop and then use the \texttt{dplyr::select()} function to remove them:

\begin{verbatim}
columns_to_drop <- c(
  "Adm0.En", "ADM0.EN..Nepal.province.shp.", "Adm0.Pcode",
  "ADM0.PCODE..Nepal.province.shp.","Adm1.En", "ADM1.EN..Nepal.province.shp.",
   "Adm1.Pcode","ADM1.PCODE..Nepal.province.shp.",
  "Adm1.Ref", "Adm1Alt1En", "Adm1Alt2En", "date..Nepal.province.shp.", "Date1",
  "Dist.Alt1E", "Dist.Alt2E", "Dist.En", "Dist.Pcode", "Dist.Ref","Unit", 
  "Valid.On", "Valid.To", "validOn..Nepal.province.shp.", 
  "validTo..Nepal.province.shp.","Area.of.Production", "Geometry", 
  "Geometry..Nepal.province.shp.","Kg.Per.Hectare", "Measure", "Metric.Ton", 
  "Number.of.Records","Shape.Area", "Shape.Area..Nepal.province.shp.", 
  "Shape.Leng", "Shape.Leng..Nepal.province.shp."
)

data_agri_clean <- data_agri %>%
  select(-all_of(columns_to_drop))
\end{verbatim}

This command creates a new cleaned version of the dataset, \texttt{data\_agri\_clean}, which contains only the essential information for further analysis and integration with the climate data.

\begin{verbatim}
data_agri_clean <- data_agri_clean %>%
  rename(Year = Date)
\end{verbatim}
\clearpage
% Fugure here----------------------------
\begin{figure}[h]
\centering
\includegraphics[width=0.6\textwidth]{figures/cleaned_agri.jpg}
\caption{ Glimpse of Agricultural Dataset after cleaning}
\end{figure}

\begin{verbatim}
summary(data_agri_clean)
\end{verbatim}

% Figure here--------------------------
\begin{figure}[h]
\centering
\includegraphics[width=0.6\textwidth]{figures/summary_agri.jpg}
\caption{ Summary of Agricultural Dataset after cleaning}
\end{figure}

When you’re working with data, it’s like putting together a giant puzzle. Sometimes, pieces of that puzzle go missing.

\subsubsection*{Analysis of the Production and Yield Variables}
Both the \texttt{Production} and \texttt{Yield} variables in the agricultural dataset exhibit missing values and considerable skewness due to extreme outliers. A careful examination of their distributions is necessary to determine the most appropriate imputation strategies.

\subsubsection*{Missing Data Overview}
\begin{itemize}
    \item \textbf{Production:} Contains 1972 missing values out of 68,243 records (approximately 2.89\%).
    \item \textbf{Yield:} Contains 1944 missing values out of 68,243 records (approximately 2.85\%).
\end{itemize}

\textbf{Summary statistics of Yield with and without outliers:}
\begin{verbatim}
summary(data_agri_clean$Yield)
\end{verbatim}
\begin{verbatim}
clean_yield <- data_agri_clean %>% 
filter(Yield >= lower_bound & Yield <= upper_bound)
summary(clean_yield$Yield)
\end{verbatim}

% Figure here----------------------------
\begin{figure}[h]
\centering
\includegraphics[width=0.6\textwidth]{figures/summary_yield.jpg}
\caption{Summary of Yield with and without Outliers}
\end{figure}

\subsubsection*{Summary statistics of Production with and without outliers:}
\begin{verbatim}
summary(data_agri_clean$Production)
\end{verbatim}

\begin{verbatim}
clean_production <- data_agri_clean %>% 
filter(Production >= lower_bound & Production <= upper_bound)
summary(clean_production$Production)
\end{verbatim}

% Figure here---------------------------
\begin{figure}[h]
\centering
\includegraphics[width=0.5\textwidth]{figures/summ_prod.jpg}
\caption{Summary of Production with and without Outliers}
\end{figure}

The analysis of both \texttt{Yield} and \texttt{Production} variables highlights significant skewness and the presence of extreme outliers in the raw data:
\begin{itemize}
    \item For \texttt{Yield}, the maximum value drops dramatically from 4,172,000.0 to 19,200.2 after outlier removal, with the mean decreasing from 5717.8 to 4361.7, and the median shifting slightly from 1689.7 to 1600.0. This confirms a strong right-skew even in the cleaned data.
    \item For \texttt{Production}, the maximum drops from 1,350,000 to 19,210, and the mean declines sharply from 7344 to 1989, while the median drops from 435 to 327. This further confirms the influence of extreme values in inflating central tendencies.
\end{itemize}

These patterns confirm that both variables are not normally distributed and are heavily affected by extreme values. The contrast between the mean and median, especially in the original data, reinforces the appropriateness of using the median as an imputation strategy for missing values. It provides a robust and representative estimate of central tendency that is not skewed by a few extreme values.
Moreover, the percentage of missing data is $2.85\%$ for Yield and $1.14\%$ for Production  is low enough to justify imputation over deletion, ensuring the retention of valuable information while maintaining the dataset’s integrity.
\textbf{Hence, median imputation will be applied to both variables using the median calculated from the dataset after outlier removal.} This approach prepares the dataset for merging with climate data in a way that is statistically sound and analytically reliable.

\subsubsection*{Median Imputation for Missing Values}

\textbf{What is Median Imputation?}\\
Median imputation is a statistical technique used to handle missing values in a dataset by replacing them with the median of the observed (non-missing) values in the same variable. The median represents the middle value of a sorted dataset and is less sensitive to extreme values (outliers) compared to the mean.

\textbf{Why Use Median Imputation?}\\
In our dataset, both \texttt{Production} and \texttt{Yield} variables exhibit strong right-skewed distributions with significant outliers. Under such conditions, using the mean for imputation would inflate the imputed values and distort the dataset. Instead, median imputation provides a more robust and representative central value.

\begin{itemize}
    \item \textbf{Resilience to Outliers:} The median is not affected by very large or very small values, making it ideal for skewed data.
    \item \textbf{Preserves Distributional Shape:} Median imputation helps maintain the original shape and spread of the variable’s distribution.
    \item \textbf{Minimal Data Loss:} This method allows us to retain all observations, ensuring the dataset remains as complete as possible.
\end{itemize}

\textbf{Application in Our Dataset}\\
We observed missing values in both variables:
\begin{itemize}
    \item \texttt{Yield:} 1944 missing values (2.85\% of total observations)
    \item \texttt{Production:} 780 missing values (1.14\% of total observations)
\end{itemize}

Due to the low proportion of missingness and the skewed distributions, we impute the missing values using the median computed from the cleaned (outlier-removed) dataset:

\begin{verbatim}
median_yield <- median(data_agri_clean$Yield, na.rm = TRUE)
data_agri_clean$Yield[is.na(data_agri_clean$Yield)] <- median_yield
sum(is.na(data_agri_clean$Yield))
summary(data_agri_clean$Yield)
\end{verbatim}

% Figure here-----------------------------
\begin{figure}[h]
\centering
\includegraphics[width=0.5\textwidth]{figures/impute_yield.jpg}
\caption{ Summary of Yield after imputation}
\end{figure}

\begin{verbatim}
median_production <- median(data_agri_clean$Production, na.rm = TRUE)
data_agri_clean$Production[is.na(data_agri_clean$Production)] <-
median_production
sum(is.na(data_agri_clean$Production))
summary(data_agri_clean$Production)
\end{verbatim}

% Figur here----------------------------
\begin{figure}[h]
\centering
\includegraphics[width=0.5\textwidth]{figures/impute_prod.jpg}
\caption{ Summary of Production after imputation}
\end{figure}


\section{Merging Agriculture Data with Climate Data}

Previously, in the data slicing chapter, we created a subset of the climate dataset that includes only the districts classified as hilly regions. In this section, we will use this filtered climate data to merge with the agriculture dataset and perform agricultural analysis focused on the hilly region.

\subsection*{Filtering Hilly Region Data}
We first create a vector containing the list of districts in the hilly region:

\begin{verbatim}
districts_to_filter <- c("Arghakhanchi", "Baglung", "Baitadi", "Bhaktapur",
"Chitwan", "Dadeldhura", "Dailekh", "Dhading","Dhankuta", "Dolpa","Gorkha",
"Gulmi", "Ilam", "Jumla", "Kabhre", "Kaski", "Kathmandu", "Lalitpur",
"Lamjung", "Makwanpur", "Myagdi", "Nuwakot","Okhaldhunga","Palpa","Parbat",
"Rukum", "Salyan", "Sindhuli", "Surkhet", "Syangja")
\end{verbatim}

Then, we filter the climate dataset to include only the records corresponding to these districts:

\begin{verbatim}
filtered_hilly_data <- subset(df_climate, District %in% districts_to_filter)
\end{verbatim}

\subsection*{Extracting the Year from the Date Column}
Before merging the datasets, we need to extract the \texttt{Year} from the \texttt{Date} column in the filtered climate data:

\begin{verbatim}
filtered_hilly_data$Year <- year(filtered_hilly_data$Date)
\end{verbatim}

\subsection*{Saving the Filtered Dataset}
After filtering and extracting the year, we save the filtered dataset as a CSV file for future use:

\begin{verbatim}
write.csv(filtered_hilly_data,row.names = FALSE,file ="dataframe_hilly.csv")
\end{verbatim}

\subsection*{Merge Criteria}
To ensure proper alignment of the datasets, we merge them based on the following common identifiers:
\begin{itemize}
    \item \textbf{District:} The administrative region where both agricultural and climate data were recorded.
    \item \textbf{Year:} The calendar year of the recorded observations.
\end{itemize}

\subsection*{Merging the Datasets}
We merge the agricultural dataset \texttt{data\_agri\_clean} with the filtered hilly region climate dataset \texttt{filtered\_hilly\_data} based on the common identifiers:

\begin{verbatim}
merged_data <- merge(data_agri_clean, filtered_hilly_data, 
by = c("District", "Year"), all = FALSE) 
\end{verbatim}

This command performs an inner join, ensuring that only records with matching \texttt{District} and \texttt{Year} from both datasets are retained.

\section{Data Analysis and Visualization}

\subsection*{Identifying Major Crops in Hilly Regions}

This section identifies the main crops grown in Nepal’s hilly districts. By merging the cleaned agriculture and climate datasets, we analyze total crop production over time. Grouping by crop and summing production across years helps highlight the most widely produced crops in these regions.


\begin{verbatim}
top_crops <- merged_data %>%
  group_by(Crop) %>%
  summarise(Total_Production = sum(Production, na.rm = TRUE)) %>%
  arrange(desc(Total_Production))
ggplot(head(top_crops, 10), aes(
  x = Total_Production,y = reorder(Crop, Total_Production) )) +
geom_col(fill = "steelblue") + coord_flip() +
labs(title = "Top 10 Crops by Total Production in Hilly Regions",
x = "Crop", y = "Total Production")
\end{verbatim}

% Figure here-----------------------------
\begin{figure}[h]
\centering
\includegraphics[width=0.5\textwidth]{figures/bar_agri.jpg}
\caption{ Top 10 Crops by Total Production in Hilly Region}
\end{figure}

The resulting plot clearly highlights the dominant crops cultivated in Nepal’s hilly regions. This insight is particularly valuable for stakeholders aiming to understand regional agricultural strengths or to design policies that support the most productive crops in these areas. Maize and Paddy are highly cultivated compared to others in hilly region.

\subparagraph*{Visualization of Agricultural Production by Crop Type in Hilly Districts
}
To understand the overall composition of agricultural production in the hilly districts, we visualize how different crop types contribute to production across various districts. This is done using a stacked bar chart, where each bar represents a district, and the stacked segments represent crop types.

\begin{verbatim}
ggplot(merged_data, aes(x = District, y = Production, fill = Crop.Type)) +
  geom_bar(stat = "identity", position = "stack") +
  coord_flip()
\end{verbatim}

% Figure here-----------------------------
\begin{figure}[h]
\centering
\includegraphics[width=0.6\textwidth]{figures/stacked_agri.jpg}
\caption{Stacked Barchart with crop type production in different districts}
\end{figure}

This visualization helps us compare which crop types dominate production in each district, and whether some regions have a more diverse agricultural profile than others. From the chart we can we that cereal production is highly dominating over hilly region.

\subsection*{Trend Analysis Over Time for Top 5 Crops}

To analyze how production of the most significant crops has changed over time in the hilly regions, we summarize and visualize the yearly production trends for the top 5 crops.

\begin{verbatim}
# Summarize yearly production for top crops
top_crops_list <- head(top_crops$Crop, 5)  # top 5 crops
yearly_trends <- merged_data %>%
  filter(Crop %in% top_crops_list) %>%
  group_by(Year, Crop) %>%
  summarise(Yearly_Production = sum(Production, na.rm = TRUE))
# Plot
ggplot(yearly_trends, aes(x = Year, y = Yearly_Production, color = Crop)) +
  geom_line(linewidth = 1) +
  labs(title = "Yearly Production Trends for Top Crops in Hilly Regions",
       x = "Year",
       y = "Production") +
  theme_minimal()
\end{verbatim}

% Figure here-----------------------------
\begin{figure}[h]
\centering
\includegraphics[width=0.5\textwidth]{figures/top5_agri.jpg}
\caption{Trend analysis of Top 5 crops in Hilly}
\end{figure}

\subsection*{Yearly Production of Paddy and Precipitation Trend}

To explore the relationship between agricultural production and climate, we analyze the yearly production of Paddy alongside average precipitation. The following R code aggregates yearly Paddy production and climate data, then plots them with dual y-axes to visualize trends concurrently.

\begin{verbatim}
# Assuming df_paddy_yearly is already created:
df_paddy_yearly <- merged_data %>%
  filter(Crop == "Paddy") %>%
  group_by(Year) %>%
  summarise(
    total_production = sum(Production, na.rm = TRUE),
    avg_temperature = mean(Temp_2m, na.rm = TRUE),
    avg_precip = mean(Precip, na.rm = TRUE)
  )


# Get the max values for dynamic scaling
max_prod_val <- max(df_paddy_yearly$total_production, na.rm = TRUE)
max_precip_val <- max(df_paddy_yearly$avg_precip, na.rm = TRUE)


target_max_proportion <- 0.7


ggplot(df_paddy_yearly, aes(x = Year)) +
  # Line and points for Production
geom_line(aes(y = total_production, color = "Total Production (Kg)"), 
linewidth = 1.2) +
geom_point(aes(y = total_production, color = "Total Production (Kg)"),
size = 3) +

# Line for Precipitation, scaled using the adjusted proportion
geom_line(aes(
  y = avg_precip * (target_max_proportion * max_prod_val / max_precip_val),
  color = "Average Precipitation (mm)"), linewidth = 1.2) +


# Add second axis for Precipitation, using the inverse of the adjusted scaling
scale_y_continuous(
  name = "Total Production (Kg)",
  sec.axis = sec_axis(~ . * (max_precip_val /
   (target_max_proportion * max_prod_val)),name = "Average Precipitation (mm)")
  ) +
# Manually set colors for the lines and define legend labels
scale_color_manual(
  name = "Variable",
  values = c(
      "Total Production (Kg)" = "darkgreen",
      "Average Precipitation (mm)" = "red"
    )
) +
# Titles and labels
labs(
    title = "Yearly Production of Paddy with Precipitation Trend",
    x = "Year"
  ) +
theme_minimal() +
theme(
    axis.text.x = element_text(angle = 45, hjust = 1),
    axis.title.y.left = element_text(color = "darkgreen"),
    axis.title.y.right = element_text(color = "red"),
    legend.position = "bottom",
    legend.title = element_blank(),
    plot.title = element_text(hjust = 0.5, face = "bold")
  )
\end{verbatim}

% Figure here-----------------------------
\begin{figure}[h]
\centering
\includegraphics[width=0.5\textwidth]{figures/paddy_trend.jpg}
\caption{Yearly Production of Paddy and Precipitation Trend}
\end{figure}

This plot illustrates how Paddy production varies over the years alongside changes in average precipitation, providing insight into the dependency of agricultural yields on climatic factors in Nepal’s hilly regions. Paddy production has steadily increased over the years, but rainfall shows high variability. Notably, drops in rainfall often correspond to declines in paddy production, highlighting the critical role of adequate rainfall for consistent harvests despite improvements in farming practices.

\subsection*{Correlation Heatmap}

To better understand how climatic factors relate to agricultural outcomes such as crop yield and production, we analyze the correlations between key climate variables and agricultural metrics. This helps reveal any direct linear relationships and the strength of connections within the climate variables themselves. The following code computes and visualizes these correlations using a heatmap.

\begin{verbatim}
library(dplyr)
library(corrplot)
# Step 1: Create a clean numeric dataset for correlation
climate_agri_subset <- aggregated_data %>%
select(avg_temperature, avg_precip, avg_humidity, avg_pressure, avg_wind,
Yield, Production)

# Step 2: Compute correlation matrix
cor_matrix <- cor(climate_agri_subset, use = "complete.obs")

# Step 3: Visualize using corrplot
corrplot(cor_matrix, method = "color",
         type = "upper",        # Only upper triangle
         tl.col = "black",      # Text label color
         addCoef.col = "black", # Add correlation values
         number.cex = 0.7,      # Size of numbers
         col = colorRampPalette(c("red", "white", "blue"))(200))
\end{verbatim}

% Figure here-----------------------------
\begin{figure}[h]
\centering
\includegraphics[width=0.5\textwidth]{figures/corr_agri.png}
\caption{Correlation Analysis}
\end{figure}

From this heatmap, several interesting points emerge:

\textbf{Weak Direct Link to Yield and Production:}  
  Correlations between Yield/Production and climate variables (temperature, precipitation, humidity, pressure, wind) are near zero, indicating little to no linear relationship at the annual average level.
  
 \textbf{Why This Matters:}  
 \begin{itemize}
  \item Linear correlation misses non-linear effects (e.g., too much or too little rain affecting yield).  
  \item Annual averages may mask critical seasonal or extreme weather impacts.  
  \item Other factors like technology or irrigation might overshadow climate effects.
 \end{itemize}

 \textbf{Strong Climate Variable Relationships:}  
 \begin{itemize}
  \item Temperature and Pressure: Nearly perfect positive correlation (0.99), suggesting redundancy.  
  \item Humidity: Positively correlated with temperature (0.78) and precipitation (0.65), reflecting natural atmospheric moisture dynamics.  
  \item Wind and Precipitation: Moderately negatively correlated (-0.51), meaning windy years tend to be drier.
 \end{itemize}




