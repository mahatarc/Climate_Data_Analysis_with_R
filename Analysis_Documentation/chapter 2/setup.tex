\chapter{Preparing Climate Data with R in Colab}

Welcome to the practical side of data analysis! In Chapter 1, we explored the power of R packages. Now, we’ll apply that knowledge to a real-world scenario: working with climate data. Climate data, like weather station readings (temperature, precipitation, etc.), is fascinating but often comes with imperfections. It might have missing measurements, incorrect values, or inconsistent formats.

Before we can analyze data and draw meaningful conclusions, we must clean and preprocess it. This means identifying and fixing errors, handling missing information, and transforming the data into a suitable format for analysis. Think of it like preparing your ingredients before cooking – essential for a good final result!

In this chapter, we’ll use Google Colaboratory (Colab), a free, cloud-based platform that lets you write and run code (including R!) directly in your web browser. This is fantastic because you don’t need to install complex software on your own computer, and you can easily save and share your work.

\section{Setting Up Your Colab Environment for R}

Google Colab primarily supports Python, but we can easily configure it to run R code. To set up R environment we can perform following steps:

\begin{enumerate}
    \item \textbf{Access Google Colab:} Open your web browser and go to \url{https://colab.research.google.com}. You’ll need a Google account to use it.
    \item \textbf{Create a New Notebook:} Click on \texttt{File} $\rightarrow$ \texttt{New notebook}. This creates a new, empty Colab notebook file (which is automatically saved to your Google Drive). Rename it something descriptive, like ``Climate Dataset Analysis.''
    \item \textbf{Understanding the Interface:}
    \begin{itemize}
        \item \textbf{Cells:} Colab notebooks are made of cells. You’ll use Code cells to write and run R code and Text cells to add notes and explanations using Markdown.
        \item \textbf{Runtime:} This is the virtual machine in the cloud running your code.
        \item \textbf{Running Cells:} To run a cell, click the play button to its left or press \texttt{Shift + Enter}.
    \end{itemize}
    \item \textbf{Change Runtime to R:}
    \begin{itemize}
        \item Click \texttt{Runtime} on the menu.
        \item Choose \texttt{Change runtime type}.
        \item In the dropdown under ``Language,'' select \texttt{R}.
        \item Click \texttt{Save}.
    \end{itemize}
    \item \textbf{Installing Packages:} As shown in the setup cell, you install R packages in Colab just like you would locally, using \texttt{install.packages("package name")}. Any packages you install will be available for your current session but might need reinstalling if the runtime fully resets after a long period of inactivity.
\end{enumerate}

\section{Getting Climate Data into Colab}

Now we need some reliable data to conduct meaningful analysis. In this chapter, we will work with climate data from Nepal, covering the period from 1980 to 2019. This dataset includes important environmental indicators such as temperature, precipitation, and humidity, which will help us explore climate trends over time.

\subsection*{Where to Find Climate Data:}

The dataset used in this chapter is publicly available on the Kaggle platform, a popular repository for datasets and data science projects. The dataset contains climate data for Nepal, including various meteorological parameters. These data were sourced from the NASA Langley Research Center (LaRC) POWER Project, which is funded through the NASA Earth Science/Applied Science Program. The data was extracted via NASA’s Power Access API. The information is based on measurements from 93 weather stations across 62 districts in Nepal, providing a comprehensive view of the country’s climate. We will import the required dataset directly from Kaggle. Make sure you have a Kaggle account.

You can view or download the dataset using the following link:

\medskip

\href{https://www.kaggle.com/code/saimondahal/nepal-climate-data-eda-insights/input}{Click here to access the dataset on Kaggle
}
\medskip

\subsection*{Loading Dataset in Colab:}

The dataset that we downloaded is in \texttt{.csv} format. We can rename the downloaded dataset as \texttt{dailyclimate.csv}. There are many ways to load the dataset in the Colab environment.
\begin{enumerate}
\item {Uploading Directly (Best for small files, $<$ 25MB)}  
    \begin{itemize}
        \item Click the folder icon in the left sidebar of Colab.
        \item Click the ``Upload'' button.
        \item Select your \texttt{weather data raw.csv} file from your computer.
    \end{itemize}
\noindent \textit{Note: Files uploaded this way disappear when the runtime restarts.}

\item {Mounting Google Drive (Recommended for larger/persistent files)}  
For this you must have the \texttt{dailyclimate.csv} in your Google Drive. Place your \texttt{dailyclimate.csv} file somewhere convenient in your Google Drive (e.g., in a folder called \texttt{Colab Data}).

    \begin{itemize}
        \item Click the folder icon in the left sidebar.
        \item Click the ``Mount Drive'' button (looks like a Google Drive logo).
        \item Follow the prompts in the popup window and the new code cell that appears. You’ll need to authorize Colab to access your Google Drive.
        \item Once mounted, your Google Drive files will appear under \texttt{/content/drive/MyDrive/}. You can navigate this like any other folder system.
    \end{itemize}
\end{enumerate}
We have successfully loaded the dataset in our Colab environment and now we can begin our data preparation for analysis.

\section{First Look - Exploring Your Climate Data}

Before cleaning, we need to understand our raw data. What does it look like? What kind of information does it contain? Are there obvious problems?

\subsection{Loading the Data into R}

Assuming you have the file available (either uploaded or on Drive), load it using \texttt{readr} (part of the tidyverse).

\begin{verbatim}
library(tidyverse)

# Path if mounted on Google Drive:
file_path <- "/content/drive/MyDrive/Colab_Data/dailyclimate.csv"

# Load the data
climate_data <- read.csv(file_path)

# Quick check
head(climate_data) # Display the first 6 rows
\end{verbatim}

\subsection{Dataset Dimension}

\begin{verbatim}
dim(climate_data)
\end{verbatim}
Dimension: 883,128 rows $\times$ 23 columns.

The dataset has 883,128 rows and 23 columns.

\subsection{Dataset Structure}

Understanding the structure of the dataset is essential before proceeding to cleaning. In R, we can use the below function to inspect the dataset.

\begin{verbatim}
str(climate_data)
\end{verbatim}

% Figure here ----------------------------
\begin{figure}[h!]
    \centering
    \includegraphics[width=0.7\textwidth]{figures/raw.png}
    \caption{Structure of the Climate Dataset.}
\end{figure}

\subsection{Data Summary}

The \texttt{summary()} function in R provides a quick statistical summary of each column in the dataset. It includes descriptive statistics like mean, median, minimum, maximum, and quartiles. It also shows missing values (NA) and category counts for factors.

\begin{verbatim}
summary(climate_data)
\end{verbatim}

% Figure here----------------------------
\begin{figure}[h!]
    \centering
     \includegraphics[width=0.5\textwidth]{figures/summary.png}
    \caption{Summary of the Climate Dataset.}
\end{figure}

\subsubsection*{What to Look for During Exploration:}

\begin{enumerate}
    \item \textbf{Data Types:} Are dates recognized as dates? Are numeric values (like temperature) stored as numbers or text?
    \item \textbf{Missing Values (NA):} How many missing values are there in each column (\texttt{summary()} is great for this)?
    \item \textbf{Ranges:} Do minimum and maximum values make sense (\texttt{summary()})? (e.g., Temperatures shouldn’t be -200°C, precipitation shouldn’t be negative).
\end{enumerate}

\section{Data Preparation}

Cleaning is a critical step in any data analysis pipeline. Think of it as preparing a field before planting .We remove the weeds, till the soil, and make sure everything is ready for growth. Similarly, we clean our dataset to remove inconsistencies and get it analysis-ready.

\subsection*{Loading Dataset}
To load the agricultural dataset into R, we use the following command:

\begin{verbatim}
data_agri <- read.csv("/content/agriculture.csv")
\end{verbatim}

\subsection*{Data Inspection}
With the agricultural dataset already loaded into \texttt{data\_agri}, our next step is to inspect its structure and identify any data quality issues. This inspection helps us understand what kind of cleaning might be necessary before merging it with the climate dataset. We use the following commands to explore the dataset:

\begin{verbatim}
glimpse(data_agri)
\end{verbatim}

% Figure here-----------------------------
\begin{figure}[h]
\centering
\includegraphics[width=0.6\textwidth]{figures/agri_glimpse.jpg}
\caption{Glimpse of Agricultural Dataset}
\end{figure}

\subsection*{Data Cleaning}
\subsubsection*{Checking for Missing Values}
The next step is checking for missing values in the dataset to understand the completeness of the data.

\begin{verbatim}
sum(is.na(data_agri))          # Total missing values
colSums(is.na(data_agri))      # Missing values per column
\end{verbatim}

% Figure here------------------------
\begin{figure}[h]
\centering
\includegraphics[width=0.8\textwidth]{figures/missing_data_agri.jpg}
\caption{Missing data information
}

\end{figure}

\textbf{Explanation of each function:}
\begin{itemize}
    \item \texttt{glimpse(data\_agri)}: Offers a compact view of the dataset, showing each column’s name, data type, and a preview of its values.
    \item \texttt{sum(is.na(data\_agri))}: Displays the total count of missing values across the entire dataset.
    \item \texttt{colSums(is.na(data\_agri))}: Shows the number of missing entries in each column individually, helping us pinpoint where problems lie.
\end{itemize}

This step is crucial for understanding the completeness and structure of the data. Based on these results, we’ll decide which cleaning operations are necessary to prepare the dataset for merging with the climate records.

\subsubsection*{Dropping Unwanted Columns}
Based on our inspection, we identified several metadata and geometry-related columns in the agricultural dataset that are not relevant for our analysis. These include administrative codes, geometry information, shape metrics, and other auxiliary fields. To streamline our dataset and retain only meaningful columns, we define a list of columns to drop and then use the \texttt{dplyr::select()} function to remove them:

\begin{verbatim}
columns_to_drop <- c(
  "Adm0.En", "ADM0.EN..Nepal.province.shp.", "Adm0.Pcode",
  "ADM0.PCODE..Nepal.province.shp.","Adm1.En", "ADM1.EN..Nepal.province.shp.",
   "Adm1.Pcode","ADM1.PCODE..Nepal.province.shp.",
  "Adm1.Ref", "Adm1Alt1En", "Adm1Alt2En", "date..Nepal.province.shp.", "Date1",
  "Dist.Alt1E", "Dist.Alt2E", "Dist.En", "Dist.Pcode", "Dist.Ref","Unit", 
  "Valid.On", "Valid.To", "validOn..Nepal.province.shp.", 
  "validTo..Nepal.province.shp.","Area.of.Production", "Geometry", 
  "Geometry..Nepal.province.shp.","Kg.Per.Hectare", "Measure", "Metric.Ton", 
  "Number.of.Records","Shape.Area", "Shape.Area..Nepal.province.shp.", 
  "Shape.Leng", "Shape.Leng..Nepal.province.shp."
)

data_agri_clean <- data_agri %>%
  select(-all_of(columns_to_drop))
\end{verbatim}

This command creates a new cleaned version of the dataset, \texttt{data\_agri\_clean}, which contains only the essential information for further analysis and integration with the climate data.

\begin{verbatim}
data_agri_clean <- data_agri_clean %>%
  rename(Year = Date)
\end{verbatim}
\clearpage
% Fugure here----------------------------
\begin{figure}[h]
\centering
\includegraphics[width=0.6\textwidth]{figures/cleaned_agri.jpg}
\caption{ Glimpse of Agricultural Dataset after cleaning}
\end{figure}

\begin{verbatim}
summary(data_agri_clean)
\end{verbatim}

% Figure here--------------------------
\begin{figure}[h]
\centering
\includegraphics[width=0.6\textwidth]{figures/summary_agri.jpg}
\caption{ Summary of Agricultural Dataset after cleaning}
\end{figure}

When you’re working with data, it’s like putting together a giant puzzle. Sometimes, pieces of that puzzle go missing.

\subsubsection*{Analysis of the Production and Yield Variables}
Both the \texttt{Production} and \texttt{Yield} variables in the agricultural dataset exhibit missing values and considerable skewness due to extreme outliers. A careful examination of their distributions is necessary to determine the most appropriate imputation strategies.

\subsubsection*{Missing Data Overview}
\begin{itemize}
    \item \textbf{Production:} Contains 1972 missing values out of 68,243 records (approximately 2.89\%).
    \item \textbf{Yield:} Contains 1944 missing values out of 68,243 records (approximately 2.85\%).
\end{itemize}

\textbf{Summary statistics of Yield with and without outliers:}
\begin{verbatim}
summary(data_agri_clean$Yield)
\end{verbatim}
\begin{verbatim}
clean_yield <- data_agri_clean %>% 
filter(Yield >= lower_bound & Yield <= upper_bound)
summary(clean_yield$Yield)
\end{verbatim}

% Figure here----------------------------
\begin{figure}[h]
\centering
\includegraphics[width=0.6\textwidth]{figures/summary_yield.jpg}
\caption{Summary of Yield with and without Outliers}
\end{figure}

\subsubsection*{Summary statistics of Production with and without outliers:}
\begin{verbatim}
summary(data_agri_clean$Production)
\end{verbatim}

\begin{verbatim}
clean_production <- data_agri_clean %>% 
filter(Production >= lower_bound & Production <= upper_bound)
summary(clean_production$Production)
\end{verbatim}

% Figure here---------------------------
\begin{figure}[h]
\centering
\includegraphics[width=0.5\textwidth]{figures/summ_prod.jpg}
\caption{Summary of Production with and without Outliers}
\end{figure}

The analysis of both \texttt{Yield} and \texttt{Production} variables highlights significant skewness and the presence of extreme outliers in the raw data:
\begin{itemize}
    \item For \texttt{Yield}, the maximum value drops dramatically from 4,172,000.0 to 19,200.2 after outlier removal, with the mean decreasing from 5717.8 to 4361.7, and the median shifting slightly from 1689.7 to 1600.0. This confirms a strong right-skew even in the cleaned data.
    \item For \texttt{Production}, the maximum drops from 1,350,000 to 19,210, and the mean declines sharply from 7344 to 1989, while the median drops from 435 to 327. This further confirms the influence of extreme values in inflating central tendencies.
\end{itemize}

These patterns confirm that both variables are not normally distributed and are heavily affected by extreme values. The contrast between the mean and median, especially in the original data, reinforces the appropriateness of using the median as an imputation strategy for missing values. It provides a robust and representative estimate of central tendency that is not skewed by a few extreme values.
Moreover, the percentage of missing data is $2.85\%$ for Yield and $1.14\%$ for Production  is low enough to justify imputation over deletion, ensuring the retention of valuable information while maintaining the dataset’s integrity.
\textbf{Hence, median imputation will be applied to both variables using the median calculated from the dataset after outlier removal.} This approach prepares the dataset for merging with climate data in a way that is statistically sound and analytically reliable.

\subsubsection*{Median Imputation for Missing Values}

\textbf{What is Median Imputation?}\\
Median imputation is a statistical technique used to handle missing values in a dataset by replacing them with the median of the observed (non-missing) values in the same variable. The median represents the middle value of a sorted dataset and is less sensitive to extreme values (outliers) compared to the mean.

\textbf{Why Use Median Imputation?}\\
In our dataset, both \texttt{Production} and \texttt{Yield} variables exhibit strong right-skewed distributions with significant outliers. Under such conditions, using the mean for imputation would inflate the imputed values and distort the dataset. Instead, median imputation provides a more robust and representative central value.

\begin{itemize}
    \item \textbf{Resilience to Outliers:} The median is not affected by very large or very small values, making it ideal for skewed data.
    \item \textbf{Preserves Distributional Shape:} Median imputation helps maintain the original shape and spread of the variable’s distribution.
    \item \textbf{Minimal Data Loss:} This method allows us to retain all observations, ensuring the dataset remains as complete as possible.
\end{itemize}

\textbf{Application in Our Dataset}\\
We observed missing values in both variables:
\begin{itemize}
    \item \texttt{Yield:} 1944 missing values (2.85\% of total observations)
    \item \texttt{Production:} 780 missing values (1.14\% of total observations)
\end{itemize}

Due to the low proportion of missingness and the skewed distributions, we impute the missing values using the median computed from the cleaned (outlier-removed) dataset:

\begin{verbatim}
median_yield <- median(data_agri_clean$Yield, na.rm = TRUE)
data_agri_clean$Yield[is.na(data_agri_clean$Yield)] <- median_yield
sum(is.na(data_agri_clean$Yield))
summary(data_agri_clean$Yield)
\end{verbatim}

% Figure here-----------------------------
\begin{figure}[h]
\centering
\includegraphics[width=0.5\textwidth]{figures/impute_yield.jpg}
\caption{ Summary of Yield after imputation}
\end{figure}

\begin{verbatim}
median_production <- median(data_agri_clean$Production, na.rm = TRUE)
data_agri_clean$Production[is.na(data_agri_clean$Production)] <-
median_production
sum(is.na(data_agri_clean$Production))
summary(data_agri_clean$Production)
\end{verbatim}

% Figur here----------------------------
\begin{figure}[h]
\centering
\includegraphics[width=0.5\textwidth]{figures/impute_prod.jpg}
\caption{ Summary of Production after imputation}
\end{figure}

